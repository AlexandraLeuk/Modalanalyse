
\pdfminorversion=4


\documentclass[conference]{IEEEtran}

\usepackage[normalem]{ulem}
\usepackage{amsmath}
\usepackage[pdftex]{graphicx}
\usepackage{cite}
\usepackage{xspace}
\usepackage{tikz}
\usepackage{pgfplots}
\usepackage[per-mode=symbol]{siunitx}
\usepackage{booktabs}
\usepackage{tikzscale}
\usepackage{array}
\usepackage{amsmath,amsfonts,amssymb}

% correct bad hyphenation here
%\hyphenation{op-tical net-works semi-conduc-tor}
\begin{document}

% paper title
% Titles are generally capitalized except for words such as a, an, and, as,
% at, but, by, for, in, nor, of, on, or, the, to and up, which are usually
% not capitalized unless they are the first or last word of the title.
% Linebreaks \\ can be used within to get better formatting as desired.
% Do not put math or special symbols in the title.
\title{Various concepts of modal analysis and model reduction methods}

% author names and affiliations
% use a multiple column layout for up to three different
% affiliations
\author{\IEEEauthorblockN{Alexandra Leukauf, Sebastian Thormann, Jan Valasek,  and Michael Zauner
 }
\IEEEauthorblockA{Institute of Mechanics and Mechatronics\\Technical University of Vienna, Austria\\
}}

% make the title area
\maketitle

% As a general rule, do not put math, special symbols or citations
% in the abstract
\begin{abstract}
This Paper is intended to give a short overview of 
\end{abstract}

% no keywords




% For peer review papers, you can put extra information on the cover
% page as needed:
% \ifCLASSOPTIONpeerreview
% \begin{center} \bfseries EDICS Category: 3-BBND \end{center}
% \fi
%
% For peerreview papers, this IEEEtran command inserts a page break and
% creates the second title. It will be ignored for other modes.
\IEEEpeerreviewmaketitle

\section{Introduction}
cdscsdc
\section{free oscillation Eigenvalue problem}
blub, Convergence criteria
\begin{equation}
\frac{\lVert \lambda_{\mathrm{k+1}}-\lambda_{\mathrm{k}} \rVert}{\lVert \lambda_{\mathrm{k+1}}\rVert}\geq \epsilon
\end{equation}
\subsection*{Vector Iteration}
\begin{equation}
\textbf{x}_\mathrm{k+1}=\frac{\textbf{Ax}_\mathrm{k}}{\lVert \textbf{Ax}_\mathrm{k} \rVert}
\end{equation}
\subsection*{Inverse Vector Iteration}
\begin{equation}
\textbf{x}_\mathrm{k+1}=\frac{\textbf{Bx}_\mathrm{k}}{\lVert \textbf{Bx}_\mathrm{k} \rVert}
\end{equation}
with
\begin{equation}
\textbf{B}=(\textbf{A}-\sigma \textbf{I})^{-1}
\end{equation}
$\mu$ is the eigenvalue of B.
\begin{equation}
\lambda=\sigma+\frac{1}{\mu}
\end{equation}
\subsection*{Rayleigh Quotient Iteration}
\begin{equation}
\sigma_{\mathrm{k}}=\frac{\textbf{x}_{\mathrm{k}}^\intercal\textbf{Ax}_{\mathrm{k}}}{\textbf{x}_{\mathrm{k}}^\intercal\textbf{x}_{\mathrm{k}}}
\end{equation}
\begin{equation}
\textbf{B}_{\mathrm{k}}=(\textbf{A}-\sigma_{\mathrm{k}} \textbf{I})^{-1}
\end{equation}
\subsection*{Gram-Schmid Process}
\begin{equation}
p(\textbf{v},\textbf{u})=\frac{\textbf{v}\textbf{u}}{\textbf{u}\textbf{u}}\textbf{u}
\end{equation}
\begin{equation}
\textbf{u}_{\mathrm{k}} = \textbf{v}_{\mathrm{k}} -\sum_{\mathrm{j=1}}^{\mathrm{k-1}}\mathrm{p}(\textbf{v}_{\mathrm{j}},\textbf{u}_{\mathrm{k}})
\end{equation}
\subsection*{QR-Algorithm}
\begin{equation}
\textbf{A}_\mathrm{k}={\textbf{Q}_\mathrm{k}}{\textbf{R}_\mathrm{k}}
\end{equation}
\begin{equation}
\textbf{A}_\mathrm{k+1}={\textbf{R}_\mathrm{k}}{\textbf{Q}_\mathrm{k}}
\end{equation}
\subsection*{Subspace Iteration}
\begin{equation}
1=1
\end{equation}






\section{Time and frequency domain}
blub
\section{Model order reduction}
blub
\section{Conclusion}
% An example of a floating figure using the graphicx package.
% Note that \label must occur AFTER (or within) \caption.
% For figures, \caption should occur after the \includegraphics.
% Note that IEEEtran v1.7 and later has special internal code that
% is designed to preserve the operation of \label within \caption
% even when the captionsoff option is in effect. However, because
% of issues like this, it may be the safest practice to put all your
% \label just after \caption rather than within \caption{}.
%
% Reminder: the "draftcls" or "draftclsnofoot", not "draft", class
% option should be used if it is desired that the figures are to be
% displayed while in draft mode.
%
%\begin{figure}[!t]
%\centering
%\includegraphics[width=2.5in]{myfigure}
% where an .eps filename suffix will be assumed under latex,
% and a .pdf suffix will be assumed for pdflatex; or what has been declared
% via \DeclareGraphicsExtensions.
%\caption{Simulation results for the network.}
%\label{fig_sim}
%\end{figure}

% Note that the IEEE typically puts floats only at the top, even when this
% results in a large percentage of a column being occupied by floats.


% An example of a double column floating figure using two subfigures.
% (The subfig.sty package must be loaded for this to work.)
% The subfigure \label commands are set within each subfloat command,
% and the \label for the overall figure must come after \caption.
% \hfil is used as a separator to get equal spacing.
% Watch out that the combined width of all the subfigures on a
% line do not exceed the text width or a line break will occur.
%
%\begin{figure*}[!t]
%\centering
%\subfloat[Case I]{\includegraphics[width=2.5in]{box}%
%\label{fig_first_case}}
%\hfil
%\subfloat[Case II]{\includegraphics[width=2.5in]{box}%
%\label{fig_second_case}}
%\caption{Simulation results for the network.}
%\label{fig_sim}
%\end{figure*}
%
% Note that often IEEE papers with subfigures do not employ subfigure
% captions (using the optional argument to \subfloat[]), but instead will
% reference/describe all of them (a), (b), etc., within the main caption.
% Be aware that for subfig.sty to generate the (a), (b), etc., subfigure
% labels, the optional argument to \subfloat must be present. If a
% subcaption is not desired, just leave its contents blank,
% e.g., \subfloat[].


% An example of a floating table. Note that, for IEEE style tables, the
% \caption command should come BEFORE the table and, given that table
% captions serve much like titles, are usually capitalized except for words
% such as a, an, and, as, at, but, by, for, in, nor, of, on, or, the, to
% and up, which are usually not capitalized unless they are the first or
% last word of the caption. Table text will default to \footnotesize as
% the IEEE normally uses this smaller font for tables.
% The \label must come after \caption as always.
%
%\begin{table}[!t]
%% increase table row spacing, adjust to taste
%\renewcommand{\arraystretch}{1.3}
% if using array.sty, it might be a good idea to tweak the value of
% \extrarowheight as needed to properly center the text within the cells
%\caption{An Example of a Table}
%\label{table_example}
%\centering
%% Some packages, such as MDW tools, offer better commands for making tables
%% than the plain LaTeX2e tabular which is used here.
%\begin{tabular}{|c||c|}
%\hline
%One & Two\\
%\hline
%Three & Four\\
%\hline
%\end{tabular}
%\end{table}


% Note that the IEEE does not put floats in the very first column
% - or typically anywhere on the first page for that matter. Also,
% in-text middle ("here") positioning is typically not used, but it
% is allowed and encouraged for Computer Society conferences (but
% not Computer Society journals). Most IEEE journals/conferences use
% top floats exclusively.
% Note that, LaTeX2e, unlike IEEE journals/conferences, places
% footnotes above bottom floats. This can be corrected via the
% \fnbelowfloat command of the stfloats package.


% use section* for acknowledgment





% trigger a \newpage just before the given reference
% number - used to balance the columns on the last page
% adjust value as needed - may need to be readjusted if
% the document is modified later
%\IEEEtriggeratref{8}
% The "triggered" command can be changed if desired:
%\IEEEtriggercmd{\enlargethispage{-5in}}

% references section

% can use a bibliography generated by BibTeX as a .bbl file
% BibTeX documentation can be easily obtained at:
% http://mirror.ctan.org/biblio/bibtex/contrib/doc/
% The IEEEtran BibTeX style support page is at:
% http://www.michaelshell.org/tex/ieeetran/bibtex/
%\bibliographystyle{IEEEtran}
% argument is your BibTeX string definitions and bibliography database(s)
%\bibliography{paper_lit}

\bibliographystyle{unsrt}
\bibliography{references}
% that's all folks
\end{document}


